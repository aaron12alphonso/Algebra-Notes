\chapter{Groups}

\textbf{Note to the reader:} This is not an exhaustive list of important results required for competitive exams but rather a starting point. The git hub link  for these notes is https://github.com/aaron12alphonso/Algebra-Notes.git                   
\section{Basic Counting Results in Groups}

\begin{theorem}
	If $G$ is a group and $x\in G$ is of order $n$ then for any $k\in \N$
	\begin{align*}
		|x^k|=\dfrac{n}{\gcd(n,k)}
	\end{align*} 	
\end{theorem}

\begin{theorem}
	Let $G$ be a group and $x\in G$ be of order $n$ and suppose $s,t\in \N$. Then $\left\langle x^s\right\rangle=\left\langle x^t\right\rangle$ if and only if $\gcd(n,s)=\gcd(n,t)$.
\end{theorem}

\begin{theorem}[Lagrange's Theorem]
	If $G$ is a finite group and $H$ is a subgroup of $G$ then $|G|=|H||G/H|$ and hence $|H|$ divides $|G|$.
\end{theorem}

\begin{theorem}
	If $H$ and $K$ are finite subsets of a group $G$ then 
	\begin{align}
		|HK|=\dfrac{|H||K|}{|H\cap K|}
	\end{align}
\end{theorem}

\begin{theorem}[Sylow's Theorem]
	Let $G$ be a group of order $p^\alpha m$ where $p$ is a prime such that $p\not\divides m$ Then 
	\begin{enumerate}
		\item 
		$G$ has a subgroup of order $p^\alpha$ i.e $G$ has a Sylow $p-$ subgroup.
		\item 
		If $P$ and $Q$ are Sylow $p-$ subgroups then $Q=gPg^{-1}$ for some $g\in G$.
		\item 
		If $n_p$ is the number of Sylow $p-$ subgroups of $G$ then $n_p\equiv 1\pmod p$ and $n_p\divides m$.
	\end{enumerate}
\end{theorem}

\begin{theorem}
	If $p$ and $q$ are primes such that $p<q$ and $q\not\equiv 1\pmod p$ then and group of order $pq$ is cyclic.
\end{theorem}

\begin{theorem}[Fundamental Theorem of Finitely Generated Abelian Groups]
	If $G$ is a finitely generated abelian group then there exists integers $r,n_1,n_2,\dots,n_s$ such that $|G|\equiv \Z^r\times \Z_{n_1}\times \Z_{n_2}\times\cdots\times \Z_{n_s}$ where
		\begin{enumerate}
			\item 
			$r\geq 0$ and $n_j\geq 0$ for $1\leq j\leq s$
			\item 
			$n_{j+1}\divides n_j$ for $1\leq j\leq s-1$		
		\end{enumerate}
\end{theorem}

\section{Homomorphisms}

\begin{theorem}
	Let $f:G\rightarrow H$ be a group homomorphism. Then we have the following properties
	\begin{itemize}
		\item 
		For any finite subgroup $S$ of $G$, $|f(S)|\divides |S|$
		\item 
		For any $a\in G$ of finite order $|f(a)|\divides |a|$
	\end{itemize}
\end{theorem}

\begin{remark}
	Notice how the above theorem can help you to compute the number of homomorphisms from some group $G$ to another group $H$.
\end{remark}

\section{The Group $\Z_n$}
\begin{theorem}
	For $n\in \N$, $\Z_n$ has the following properties
	\begin{itemize}
		\item 
		The group $\Z_n$ has exactly $\phi(n)$ generators.
		\item 
		The number of subgroups of $\Z_n$ are the number of divisors of $n$. 
		\item 
		For $a\in \Z_n$, $|a|=\dfrac{n}{\gcd(a,n)}$
		\item 
		For $a,b\in \Z_n$, $<a>=<b>$ if and only if $\gcd(a,n)=\gcd(a,n)$.
	\end{itemize}
\end{theorem}

\begin{theorem}
	The number of homomorphisms from $\Z_n$ to $\Z_m$ is $\gcd(n,m)$.	
\end{theorem}

\section{The Group $S_n$}
-The group of all permutations of an $n-$ set
\begin{theorem}[Properties of $S_n$]
	\begin{enumerate}
		\item[]
		\item 
		The set of all transpositions is a generating set of $S_n$
		\item 
		For $\sigma=(a_1,a_2,\dots,a_k)\in S_n$ and $\tau\in S_n$, $\tau\sigma\tau^{-1}=(\tau(a_1),\tau(a_2),\dots,\tau(a_k))$.
		\item 
		The for any $n-$ cycle $\sigma$ and any transposition $\tau$ in $S_n$ the set $\{\sigma,\tau\}$ generates $S_n$.  
		\item 
		The number of conjugacy classes of $S_n$ is the number of partitions of $n$, $p(n)$. 
		\item 
		The order of a permutation in $S_n$ is the $lcm$ of the lengths of it's cycles when the permutation is written as a product of disjoint cycles. 
	\end{enumerate}
\end{theorem}

\section{The Group $D_{2n}/D_n$}
- The group of all symmetries of a regular polygon of $n-$ vertices.
\begin{theorem}[Properties of $D_n$]
	\begin{enumerate}
		\item[]
		\item 
		$D_{2n}$ consist of $2n$ elements, $n$ of which are rotations and the rest are reflections
		\item 
		If $s$ is the reflection of the regular $n-$ gon about the line of symmetry that passes through the vertex $1$ and the origin and suppose $r$ is the rotation of the regular $n-$ gon by $2\pi/n$ radians then $D_{2n}=\{sr^k:0\leq k\leq n-1\}$
		\item 
		The composition of two rotations is a rotation.
		\item 
		The composition of a rotations and a reflection is a reflection.
		\item 
		The composition of two reflections is a rotation.
		\item 
		The inverse of a rotation is a rotation.
		\item 
		The inverse of a reflection is the same reflection. That is reflections are order $2$ elements of $D_{2n}$
	\end{enumerate}
\end{theorem}

\section{The Group $A_n$}
- Subgroup of all even permutations of an $n-$ set

\begin{theorem}[Properties of $A_n$]
	\item 
	$|A_n|=n!/2$ and hence $A_n$ is a normal subgroup of $S_n$.
	\item 
	For any subgroup $H$ of $A_n$ either $H\subset A_n$ or $|H\cap A_n|=|H|/2$.
	\item 
	The set $\{(1,2,3),(1,2,4),\dots,(1,2,n)\}$ is a generating set of $A_n$.
\end{theorem}



